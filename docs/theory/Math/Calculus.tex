\hypertarget{calculus}{%
\section{Calculus}\label{calculus}}

\hypertarget{fundamentals}{%
\subsection{Fundamentals}\label{fundamentals}}

\(\frac{d}{d x} \int_{a}^{x} f(t) d t=f(x)\)

\hypertarget{differentiation}{%
\subsection{Differentiation}\label{differentiation}}

In the case of a vector field, the directional derivative is also a
vector each of whose components gives the rate of change of the
corresponding component of v in the direction of h. The gradient in this
case will be a tensor field (that when applied to h gives the
directional derivative of v in the direction of h).

\[\mathbf{e}_i = \frac{\partial}{\partial x^i} = \partial_i,\quad i = 1,\, 2,\, \dots,\, n\]

define what is referred to as the local {[}{[}basis of a vector
space\textbar basis{]}{]} of the tangent space to
\{\{math\textbar{}`'M''\}\} at each point of its domain. These can be
used to define the {[}{[}metric tensor{]}{]}:
\[g_{ij} = \mathbf{e}_i \cdot \mathbf{e}_j\]

and its inverse:

\[g^{ij} = \left( g^{-1} \right)_{ij}\]

which can in turn be used to define the dual basis:

\[\mathbf{e}^i = \mathbf{e}_j g^{ji},\quad i = 1,\, 2,\, \dots,\, n\]

Some texts write \(\mathbf{g}_i\) for \(\mathbf{e}_i\), so that the
metric tensor takes the particularly beguiling form
\(g_{ij} = \mathbf{g}_i \cdot \mathbf{g}_j\). This convention also
leaves use of the symbol \(e_i\) unambiguously for the
{[}{[}vierbein{]}{]}.

\hypertarget{gradient}{%
\subparagraph{Gradient}\label{gradient}}

\[ \begin{array}{c}{\mathbf{e}_{i} \cdot \operatorname{grad} \varphi(\mathbf{x})=[\operatorname{grad} \varphi(\mathbf{x})]_{i}=\frac{\partial \varphi(\mathbf{x})}{\partial x_{i}}}\\{\mathbf{e}_{i} \cdot \operatorname{grad} \mathbf{v}(\mathbf{x}) \mathbf{e}_{j}=[\operatorname{grad} \mathbf{v}(\mathbf{x})]_{i j}=\frac{\partial v_{i}(\mathbf{x})}{\partial x_{j}}}\end{array}\]

\hypertarget{jacobian}{%
\subparagraph{Jacobian}\label{jacobian}}

In general, the \emph{derivative} of a function
\(f : \mathbb{R}^n \to \mathbb{R}^m\) at a point \(p \in \mathbb{R}^n\),
if it exists, is the unique linear transformation
\(Df(p) \in L(\mathbb{R}^n,\mathbb{R}^m)\) such that \[
 \lim_{h \to 0} \frac{\|f(p+h)-f(p)-Df(p)h\|}{\|h\|} = 0;
\] the matrix of \(Df(p)\) with respect to the standard orthonormal
bases of \(\mathbb{R}^n\) and \(\mathbb{R}^m\), called the
\emph{Jacobian matrix} of \(f\) at \(p\), therefore lies in
\(M_{m \times n}(\mathbb{R})\).

Now, suppose that \(m=1\), so that \(f : \mathbb{R}^n \to \mathbb{R}\).
Then if \(f\) is differentiable at \(p\),
\(Df(p) \in L(\mathbb{R}^n,\mathbb{R}) = (\mathbb{R}^n)^\ast\) is a
functional, and hence the Jacobian matrix, as you point out, lies in
\(M_{1 \times n}(\mathbb{R})\), i.e., is a row vector. However, by the
Riesz representation theorem, \(\mathbb{R}^n \cong (\mathbb{R}^n)^\ast\)
via the map that sends a vector \(x \in \mathbb{R}^n\) to the functional
\(y \mapsto \left\langle y,x \right\rangle\). Hence, if \(f\) is
differentiable at \(p\), then the \emph{gradient} of \(f\) at \(p\) is
the unique (column!) vector \(\nabla f(p) \in \mathbb{R}^n\) such that
\[
 \forall h \in \mathbb{R}^n, \quad Df(p)h = \left\langle \nabla f(p),h\right\rangle;
\] in particular, if you unpack definitions, you'll find that the
Jacobian matrix of \(f\) at \(p\) is precisely \(\nabla f(p)^T\).

The Jacobian determinant can be viewed as the ratio of an infinitesimal
change in the variables of one coordinate system to another. This
requires that the function \(f(x)\) maps \(\mathbb{R}^n→\mathbb{R}^n\),
which produces an \(n×n\) square matrix for the Jacobian. For example:
\[\iiint_{R} f(x, y, z) d x d y d z=\iiint_{S} f(x(u, v, w), y(u, v, w), z(u, v, w))\left|\frac{\partial(x, y, z)}{\partial(u, v, w)}\right| d u d v d w\]

\hypertarget{directional-derivative}{%
\subparagraph{Directional Derivative}\label{directional-derivative}}

\[\operatorname{grad} \varphi(\mathbf{x})[\mathbf{h}]=\left.\frac{d}{d \alpha} \varphi(\mathbf{x}+\alpha \mathbf{h})\right|_{\alpha=0}\]

\hypertarget{divergence---curl--laplacian}{%
\subparagraph{Divergence - Curl-
Laplacian}\label{divergence---curl--laplacian}}

\[\begin{aligned} \operatorname{div} \mathbf{v}=\operatorname{tr}[\operatorname{grad} \mathbf{v}] &=\frac{\partial v_{i}}{\partial x_{i}} \\(\operatorname{div} \mathbf{T})_{i} &=\frac{\partial T_{i j}}{\partial x_{j}} \\(\operatorname{curl} \mathbf{v})_{i} &=e_{i j k} \frac{\partial v_{k}}{\partial x_{j}} \\(\operatorname{curl} \mathbf{T})_{i j} &=e_{i p q} \frac{\partial T_{j q}}{\partial x_{p}} \end{aligned}\]

\[\Delta \mathbf{v}=\operatorname{div} \operatorname{grad} \mathbf{v}, \quad \Delta v_{i}=\frac{\partial^{2} v_{i}}{\partial x_{j} \partial x_{j}}\\\]
\[\Delta T_{i j}=\frac{\partial T_{i j}}{\partial x_{k} \partial x_{k}}\]

\hypertarget{integration}{%
\subsection{Integration}\label{integration}}

\hypertarget{integration-by-parts}{%
\subparagraph{Integration by parts}\label{integration-by-parts}}

If \(u = u(x)\) and \(du = u'(x) dx\), while \(v = v(x)\) and
\(dv = v'(x) dx\), then integration by parts states that:

\[\begin{aligned}
\int_{a}^{b} u(x) v^{\prime}(x) d x &=[u(x) v(x)]_{a}^{b}-\int_{a}^{b} u^{\prime}(x) v(x) d x \\
&=u(b) v(b)-u(a) v(a)-\int_{a}^{b} u^{\prime}(x) v(x) d x
\end{aligned}\]

\hypertarget{divergence-theorem}{%
\subparagraph{Divergence Theorem}\label{divergence-theorem}}

\(\begin{array}{rl}{\int \varphi n_{i} d a} & {=\int_{R} \frac{\partial \varphi}{\partial x_{i}} d v} \\ {\int_{\partial R} v_{i} n_{i} d a} & {=\int_{R} \frac{\partial v_{i}}{\partial x_{i}} d v} \\ {\int_{\partial R} T_{i j} n_{j} d a} & {=\int_{R} \frac{\partial T_{i j}}{\partial x_{j}} d v} \\ {\partial R} & {R}\end{array}\)
