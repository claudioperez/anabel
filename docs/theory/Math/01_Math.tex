\hypertarget{i-math}{%
\section{I Math}\label{i-math}}

\hypertarget{calculus}{%
\subsection{Calculus}\label{calculus}}

\(\frac{d}{d x} \int_{a}^{x} f(t) d t=f(x)\)

\hypertarget{calculus-of-variations}{%
\subsection{Calculus of Variations}\label{calculus-of-variations}}

\emph{First variation} of a functional, \(I\):
\(\left.\delta I\{u ; w\} \stackrel{\text { def }}{=} \frac{d}{d \zeta} I\{u(x)+\zeta w(x)\}\right|_{\zeta=0}\)

The condition: \(\delta I\{u ; w\}=0\) for all admissible \(w\) is a
necessary condition for \(u(x)\) to be a minimizer of \(I\).

\textbf{Euler-Lagrance Equation}
\[\frac{\partial F}{\partial u}-\frac{d}{d x}\left(\frac{\partial F}{\partial u^{\prime}}\right)=0 \quad \forall x \in\left(x_{0}, x_{1}\right)\]

\textbf{Taking variations}

\(\delta I=\left.\frac{d}{d \zeta} I(\Psi+\zeta \delta \Psi)\right|_{\zeta=0}\)

\(\delta F=\left(\frac{\partial F}{\partial u}\right) \delta u+\left(\frac{\partial F}{\partial u^{\prime}}\right) \delta u^{\prime}\)

\(\begin{aligned} \delta I & \equiv \int_{x_{0}}^{x_{1}} \delta F\left(x, u, u^{\prime}\right) d x \\ &=\int_{x_{0}}^{x_{1}}\left(\frac{\partial F}{\partial u} \delta u+\frac{\partial F}{\partial u^{\prime}} \delta u^{\prime}\right) d x \\ &=\left.\frac{\partial F}{\partial u^{\prime}} \delta u\right|_{x_{0}} ^{x_{1}}+\int_{x_{0}}^{x_{1}}\left[\frac{\partial F}{\partial u}-\frac{d}{d x}\left(\frac{\partial F}{\partial u^{\prime}}\right)\right] \delta u d x \end{aligned}\)

\(\left.\delta H \equiv \delta\left(u^{\prime}\right) \stackrel{\text { def }}{=} \frac{d}{d \zeta} H\{u+\zeta \delta u\}\right|_{\zeta=0}=\left.\frac{d}{d \zeta}\left(u^{\prime}+\zeta(\delta u)^{\prime}\right)\right|_{\zeta=0}=(\delta u)^{\prime}\)

\hypertarget{tensors}{%
\subsection{Tensors}\label{tensors}}

\hypertarget{algebra}{%
\subsubsection{1 Algebra}\label{algebra}}

\hypertarget{tensors-1}{%
\paragraph{1.2 Tensors}\label{tensors-1}}

\(S_{i j} \stackrel{\text { def }}{=} \mathbf{e}_{i} \cdot \mathbf{S} \mathbf{e}_{j}\)

\(C_{i j k l} \stackrel{\text { def }}{=}\left(\mathbf{e}_{i} \otimes \mathbf{e}_{j}\right): \mathbb{C}\left(\mathbf{e}_{k} \otimes \mathbf{e}_{l}\right)\\\)

\(\mathbf{S}=S_{i j} \mathbf{e}_{i} \otimes \mathbf{e}_{j}\)

\(v_i =S_{i j} u_{j}\)

\(\mathbb{I}^{\mathrm{sym}} \rightarrow \frac{1}{2}\left(\delta_{i k} \delta_{j l}+\delta_{i l} \delta_{j k}\right)\)

\hypertarget{products}{%
\subparagraph{Products}\label{products}}

1.2.3 Outer Product

\[(\mathbf{u} \otimes \mathbf{v}) \mathbf{w}=(\mathbf{v} \cdot \mathbf{w}) \mathbf{u}\]

1.2.8

\[(\mathbf{S T})_{i j}=S_{i k} T_{k j}\]

1.2.11 Inner / Dot Products

\[\mathbf{u} \cdot \mathbf{v}=\mathbf{v} \cdot \mathbf{u}=u_{i} v_{j} \delta_{ij} =u_{i} v_{i}\\\mathbf{S}: \mathbf{T}=\mathbf{T}: \mathbf{S}=S_{i j} T_{i j}\\\mathbf{S}\cdot \mathbf{T}=S_{i j} T_{j l}\mathrm{e}_{i}\mathrm{e}_{l}\neq \mathbf{T}\cdot \mathbf{S}\]

1.2.1 Cross Product

\[(\mathbf{u} \times \mathbf{v})_{i}=e_{i j k} u_{j} v_{k}\]

\hypertarget{symmetric---skew-components}{%
\subparagraph{6 Symmetric - Skew
components}\label{symmetric---skew-components}}

\[\begin{array}{c}{\mathbf{S}=\mathbf{S}^{\top}, \quad S_{i j}=S_{j i}} \\ {\mathbf{\Omega}=-\mathbf{\Omega}^{\top}, \quad \Omega_{i j}=-\Omega_{j i}}\end{array}\]

\[\begin{array}{l}{(\operatorname{sym} \mathbf{T})_{i j}=\frac{1}{2}\left(T_{i j}+T_{j i}\right)} \\ {(\operatorname{skw} \mathbf{T})_{i j}=\frac{1}{2}\left(T_{i j}-T_{j i}\right)}\end{array}\]

\hypertarget{inner-product-norm}{%
\subparagraph{11 - Inner Product \& Norm}\label{inner-product-norm}}

\[|\mathbf{u}|=\sqrt{\mathbf{u} \cdot \mathbf{u}}=\sqrt{u_{i} u_{i}}\\|\mathbf{S}|=\sqrt{\mathbf{S}: \mathbf{S}}=\sqrt{S_{i j} S_{i j}}\]

\hypertarget{orthogonal-tensors}{%
\subparagraph{16 - Orthogonal tensors}\label{orthogonal-tensors}}

\(\begin{array}{c}{\mathbf{Q}^{\top} \mathbf{Q}=\mathbf{Q} \mathbf{Q}^{\top}=\mathbf{1}} \\ {\operatorname{det} \mathbf{Q}=\pm 1}\end{array}\)

\hypertarget{transformations}{%
\subparagraph{17 - Transformations}\label{transformations}}

\[v_{i}^{*}=\mathrm{e}_{i}^{*} \cdot \mathrm{v} \quad \text { and } \quad S_{i j}^{*}=\mathrm{e}_{i}^{*} \cdot \mathrm{Se}_{j}^{*}\]

\[\mathrm{Q} \stackrel{\text { def }}{=}  \mathrm{e}_{k} \otimes \mathrm{e}_{k}^{*}\]

\[v_{i}^{*}=Q_{i j} v_{j}\\S_{i j}^{*}=Q_{i k} Q_{j l} S_{k l}
\\C_{i j k l}^{*}=Q_{i p} Q_{j q} Q_{k r} Q_{l s} C_{p q r s}
\]

\hypertarget{eigenvalues}{%
\subparagraph{18 - Eigenvalues}\label{eigenvalues}}

\[\omega^{3}-I_{1}(\mathbf{S}) \omega^{2}+I_{2}(\mathbf{S}) \omega-I_{3}(\mathbf{S})=0\]
Principal Invariants:
\[\begin{array}{l}{I_{1}(\mathbf{S})=\operatorname{tr} \mathbf{S}}  = \omega_1 + \omega_2 + \omega_3 \\ {I_{2}(\mathbf{S})=\frac{1}{2}\left[(\operatorname{tr}(\mathbf{S}))^{2}-\operatorname{tr}\left(\mathbf{S}^{2}\right)\right]}  = \omega_1 \omega_2 + \omega_2 \omega_3+ \omega_3 \omega_1 \\ {I_{3}(\mathbf{S})=\operatorname{det} \mathbf{S}} = \omega_1 \omega_2 \omega_3 \end{array}\]

\hypertarget{analysis}{%
\subsubsection{2 Analysis}\label{analysis}}

\hypertarget{derivatives}{%
\paragraph{Derivatives}\label{derivatives}}

In the case of a vector field, the directional derivative is also a
vector each of whose components gives the rate of change of the
corresponding component of v in the direction of h. The gradient in this
case will be a tensor field (that when applied to h gives the
directional derivative of v in the direction of h).

Given a {[}{[}coordinate system{]}{]} \{\{math\textbar{}`'x'`'`i'`\}\}
for \{\{math\textbar{}'`i'' \{\{=\}\} 1, 2, \ldots, `'n'`\}\} on an
\{\{math\textbar{}'`n'`\}\}-manifold \{\{math\textbar{}'`M''\}\}, the
{[}{[}tangent space\textbar tangent vectors{]}{]} :\mathbf{e}\_i =
\frac{\partial}{\partial x^i} = \partial\_i,\quad i = 1,, 2,, \dots,, n

define what is referred to as the local {[}{[}basis of a vector
space\textbar basis{]}{]} of the tangent space to
\{\{math\textbar{}`'M''\}\} at each point of its domain. These can be
used to define the {[}{[}metric tensor{]}{]}: :g\_\{ij\} = \mathbf{e}\_i
\cdot \mathbf{e}\_j

and its inverse:

:g\^{}\{ij\} = \left( g\^{}\{-1\} \right)\_\{ij\}

which can in turn be used to define the dual basis:

\[\mathbf{e}^i = \mathbf{e}_j g^{ji},\quad i = 1,\, 2,\, \dots,\, n\]

Some texts write \mathbf{g}\_i for \mathbf{e}\emph{i, so that the metric
tensor takes the particularly beguiling form g}\{ij\} = \mathbf{g}\_i
\cdot \mathbf{g}\_j. This convention also leaves use of the symbol e\_i
unambiguously for the {[}{[}vierbein{]}{]}.

\hypertarget{gradient}{%
\subparagraph{Gradient}\label{gradient}}

\[ \begin{array}{c}{\mathbf{e}_{i} \cdot \operatorname{grad} \varphi(\mathbf{x})=[\operatorname{grad} \varphi(\mathbf{x})]_{i}=\frac{\partial \varphi(\mathbf{x})}{\partial x_{i}}}\\{\mathbf{e}_{i} \cdot \operatorname{grad} \mathbf{v}(\mathbf{x}) \mathbf{e}_{j}=[\operatorname{grad} \mathbf{v}(\mathbf{x})]_{i j}=\frac{\partial v_{i}(\mathbf{x})}{\partial x_{j}}}\end{array}\]

\hypertarget{jacobian}{%
\subparagraph{Jacobian}\label{jacobian}}

In general, the \emph{derivative} of a function
\(f : \mathbb{R}^n \to \mathbb{R}^m\) at a point \(p \in \mathbb{R}^n\),
if it exists, is the unique linear transformation
\(Df(p) \in L(\mathbb{R}^n,\mathbb{R}^m)\) such that \[
 \lim_{h \to 0} \frac{\|f(p+h)-f(p)-Df(p)h\|}{\|h\|} = 0;
\] the matrix of \(Df(p)\) with respect to the standard orthonormal
bases of \(\mathbb{R}^n\) and \(\mathbb{R}^m\), called the
\emph{Jacobian matrix} of \(f\) at \(p\), therefore lies in
\(M_{m \times n}(\mathbb{R})\).

Now, suppose that \(m=1\), so that \(f : \mathbb{R}^n \to \mathbb{R}\).
Then if \(f\) is differentiable at \(p\),
\(Df(p) \in L(\mathbb{R}^n,\mathbb{R}) = (\mathbb{R}^n)^\ast\) is a
functional, and hence the Jacobian matrix, as you point out, lies in
\(M_{1 \times n}(\mathbb{R})\), i.e., is a row vector. However, by the
Riesz representation theorem, \(\mathbb{R}^n \cong (\mathbb{R}^n)^\ast\)
via the map that sends a vector \(x \in \mathbb{R}^n\) to the functional
\(y \mapsto \left\langle y,x \right\rangle\). Hence, if \(f\) is
differentiable at \(p\), then the \emph{gradient} of \(f\) at \(p\) is
the unique (column!) vector \(\nabla f(p) \in \mathbb{R}^n\) such that
\[
 \forall h \in \mathbb{R}^n, \quad Df(p)h = \left\langle \nabla f(p),h\right\rangle;
\] in particular, if you unpack definitions, you'll find that the
Jacobian matrix of \(f\) at \(p\) is precisely \(\nabla f(p)^T\).

The Jacobian determinant can be viewed as the ratio of an infinitesimal
change in the variables of one coordinate system to another. This
requires that the function \(f(x)\) maps \(\mathbb{R}^n→\mathbb{R}^n\),
which produces an \(n×n\) square matrix for the Jacobian. For example:
\[\iiint_{R} f(x, y, z) d x d y d z=\iiint_{S} f(x(u, v, w), y(u, v, w), z(u, v, w))\left|\frac{\partial(x, y, z)}{\partial(u, v, w)}\right| d u d v d w\]

\hypertarget{directional-derivative}{%
\subparagraph{Directional Derivative}\label{directional-derivative}}

\[\operatorname{grad} \varphi(\mathbf{x})[\mathbf{h}]=\left.\frac{d}{d \alpha} \varphi(\mathbf{x}+\alpha \mathbf{h})\right|_{\alpha=0}\]

\hypertarget{divergence---curl--laplacian}{%
\subparagraph{Divergence - Curl-
Laplacian}\label{divergence---curl--laplacian}}

\[\begin{aligned} \operatorname{div} \mathbf{v}=\operatorname{tr}[\operatorname{grad} \mathbf{v}] &=\frac{\partial v_{i}}{\partial x_{i}} \\(\operatorname{div} \mathbf{T})_{i} &=\frac{\partial T_{i j}}{\partial x_{j}} \\(\operatorname{curl} \mathbf{v})_{i} &=e_{i j k} \frac{\partial v_{k}}{\partial x_{j}} \\(\operatorname{curl} \mathbf{T})_{i j} &=e_{i p q} \frac{\partial T_{j q}}{\partial x_{p}} \end{aligned}\]

\[\Delta \mathbf{v}=\operatorname{div} \operatorname{grad} \mathbf{v}, \quad \Delta v_{i}=\frac{\partial^{2} v_{i}}{\partial x_{j} \partial x_{j}}\\\]
\[\Delta T_{i j}=\frac{\partial T_{i j}}{\partial x_{k} \partial x_{k}}\]

\hypertarget{integration}{%
\paragraph{Integration}\label{integration}}

\hypertarget{integration-by-parts}{%
\subparagraph{Integration by parts}\label{integration-by-parts}}

If \(u = u(x)\) and \(du = u'(x) dx\), while \(v = v(x)\) and
\(dv = v'(x) dx\), then integration by parts states that:

\[\begin{aligned}
\int_{a}^{b} u(x) v^{\prime}(x) d x &=[u(x) v(x)]_{a}^{b}-\int_{a}^{b} u^{\prime}(x) v(x) d x \\
&=u(b) v(b)-u(a) v(a)-\int_{a}^{b} u^{\prime}(x) v(x) d x
\end{aligned}\]

\hypertarget{divergence-theorem}{%
\subparagraph{Divergence Theorem}\label{divergence-theorem}}

\(\begin{array}{rl}{\int \varphi n_{i} d a} & {=\int_{R} \frac{\partial \varphi}{\partial x_{i}} d v} \\ {\int_{\partial R} v_{i} n_{i} d a} & {=\int_{R} \frac{\partial v_{i}}{\partial x_{i}} d v} \\ {\int_{\partial R} T_{i j} n_{j} d a} & {=\int_{R} \frac{\partial T_{i j}}{\partial x_{j}} d v} \\ {\partial R} & {R}\end{array}\)

\hypertarget{step-functions}{%
\subsection{Step functions}\label{step-functions}}

\hypertarget{heaviside-step-function-ht}{%
\subsubsection{\texorpdfstring{Heaviside Step Function,
\(h(t)\)}{Heaviside Step Function, h(t)}}\label{heaviside-step-function-ht}}

\[h(t)=\left\{\begin{array}{l}{0 \text { for } t \leq 0} \\ {1 \text { for } t>0}\end{array}\right.\]

\hypertarget{dirac-delta-delta}{%
\subsubsection{\texorpdfstring{Dirac Delta,
\(\delta\)}{Dirac Delta, \textbackslash delta}}\label{dirac-delta-delta}}

\[\delta(t) \equiv \dot{h}(t)\]

\[\delta(t)=\left\{\begin{array}{ll}{0} & {\text { for } t \neq 0,} \\ {\infty} & {\text { for } t=0,}\end{array} \quad \text { where } \quad \int_{-\infty}^{\infty} \delta(t) d t=\int_{0^{-}}^{0^{+}} \delta(t) d t=1\right.\]

For any function continuous at \(t=0\) :

\[\int_{-\infty}^{\infty} g(t) \delta(t) d t=\int_{0^{-}}^{0^{+}} g(t) \delta(t) d t=g(0)\]

\hypertarget{convolution}{%
\subsection{Convolution}\label{convolution}}

\[
\int_{0^{-}}^{t} f(t-\tau) g(\tau) d \tau \equiv(f * g)(t)
\]

\hypertarget{laplace-transformations}{%
\subsection{Laplace Transformations}\label{laplace-transformations}}

\[
L[f(t)]=\int_{0^{-}}^{\infty} e^{-s t} f(t) d t \equiv \bar{f}(s)
\]

The laplace transformation has the following properties:

\begin{enumerate}
\def\labelenumi{\arabic{enumi}.}
\item
  \(L[(f * g)(t)]=L[f(t)] L[g(t)]=\bar{f}(s) \bar{g}(s)\)
\item
  \(L[\dot{f}(t)]=s L[f(t)]-f\left(0^{-}\right)=s \bar{f}(s)-f\left(0^{-}\right)\)
\end{enumerate}
