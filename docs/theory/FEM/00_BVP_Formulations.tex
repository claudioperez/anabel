\hypertarget{boundary-value-problems}{%
\section{Boundary Value Problems}\label{boundary-value-problems}}

\hypertarget{weak-form}{%
\subsection{Weak form}\label{weak-form}}

\begin{enumerate}
\def\labelenumi{\arabic{enumi}.}
\tightlist
\item
  Multiply both sides of the DE or PDE by an arbitrary function

  \begin{enumerate}
  \def\labelenumii{\arabic{enumii}.}
  \tightlist
  \item
    The function must be homogeneous (= 0) where displacement BC's are
    specified.
  \item
    The function must have sufficient continuity for differentiation.
  \end{enumerate}
\item
  Integrate over the domain, e.g.~length of the rod
\item
  Integrate by parts using Green's theorem to reduce derivatives to
  their minimum order.
\item
  Replace the boundary conditions by an appropriate construction.
\end{enumerate}

Note:
\begin{enumerate}
\def\labelenumi{\arabic{enumi}.}
\tightlist
\item
  Weak form is formulated in terms of axial force (or equivalently axial
  stress as in general mechanics problems). This only involves
  equilibrium
\item
  Compatibility relationship is based on infinitesimal strain, .
\item
  We did not make statements @ stress strain relationships , so the
  relationship between the strong and weak forms of equilibrium are true
  for linear \& nonlinear materials
\item
  The weak form (or the structural analysis interpretation as the PVD)
  provides a framework for finding an approximate solution
\item
  For the exact solution, we need to look at all possible trail
  functions (e.g.~describing the axial force and VD), which is a
  formidable task.
\item
  Rayleigh's method (and its extension to use superposition of several
  functions, i.e.~the Rayleigh Ritz procedure) provides a convenient way
  to limit the number of functions that we are examining and since the
  limited functions may not include the exact solution, the obtained
  solution will be an approximation
\item
  We would like to pick simple functions for easy integration \& provide
  a set of algebraic equations that can be solved efficiently. The FEM
  provides a systematic way for this.
\end{enumerate}
