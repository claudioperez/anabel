\hypertarget{govindjee}{%
\section{Govindjee}\label{govindjee}}

\hypertarget{construction}{%
\subsection{Construction}\label{construction}}

First, the solution space and the weighting function space are replaced
by finite dimensional subspaces. Second, these subspaces are employed in
the weak form to generate matrix equations which are then used to solve
for the unknown displacements at the nodes.

The space of admissible variations is defined as \[
\mathcal{V}^{h}=\left\{\boldsymbol{v}^{h}(\boldsymbol{x}) | v_{i}^{h}(\boldsymbol{x})=\sum_{A \in \eta-\eta_{u}} N_{A}(\boldsymbol{x}) v_{i A}^{h}\right\}
\] where \(\eta\) is the set of all node numbers, \(\eta_{u}\) is the
set of node numbers where the displacements are prescribed, and
\(v_{i A}^{h}\) refers to the vector \(v_{i}^{h}\) at node \(A .\)

\[
\left(\begin{array}{l}
v_{1}(\boldsymbol{x}) \\
v_{2}(\boldsymbol{x})
\end{array}\right)=\sum_{A \in \eta-\eta_{u}} N_{A}(\boldsymbol{x}) v_{i A}=\left[\begin{array}{ccccccc}
N_{1} & 0 & N_{2} & 0 & \cdots & N_{n} & 0 \\
0 & N_{1} & 0 & N_{2} & \cdots & 0 & N_{n}
\end{array}\right]\left(\begin{array}{c}
v_{11} \\
v_{21} \\
v_{12} \\
v_{22} \\
\vdots \\
v_{1 n} \\
v_{2 n}
\end{array}\right)=N \boldsymbol{v}
\] where \(n\) is the number of nodes in \(\eta-\eta_{u} .\) The \(N\)
matrix is of great utility when working with the FEM equations; as given
above, it is strictly for \(2-\mathrm{D}\) problems but is trivially
extended to 3 -D. This type of notation also extends to the symmetric
gradients of functions; viz.~in \(2-\mathrm{D}\) we have that \[
\left(\begin{array}{c}
v_{1,1} \\
v_{2,2} \\
v_{1,2}+v_{2,1}
\end{array}\right)=\left[\begin{array}{ccccccc}
N_{1,1} & 0 & N_{2,1} & 0 & \ldots & N_{n, 1} & 0 \\
0 & N_{1,2} & 0 & N_{2,2} & \ldots & 0 & N_{n, 2} \\
N_{1,2} & N_{1,1} & N_{2,2} & N_{2,1} & \ldots & N_{n, 2} & N_{n, 1}
\end{array}\right]\left(\begin{array}{c}
v_{11} \\
v_{21} \\
v_{12} \\
v_{22} \\
\vdots \\
v_{1 n} \\
v_{2 n}
\end{array}\right)=B v
\]

\hypertarget{fundamentals}{%
\subsection{Fundamentals}\label{fundamentals}}

\[\int_{\Omega}(\boldsymbol{B} \boldsymbol{v})^{T} \boldsymbol{D} \boldsymbol{B} \boldsymbol{u}=\int_{\Omega}(\boldsymbol{N} \boldsymbol{v})^{T} \boldsymbol{b}+\int_{\Gamma_{t}}(\boldsymbol{N} \boldsymbol{v})^{T} \bar{t}\]

\[k_{a b}^{e}=\int_{\Omega_{e}} \nabla N_{a}^{e} \cdot \kappa \nabla N_{b}^{e}=\int_{\Omega_{e}} N_{a, i}^{e} \kappa_{i j} N_{b, j}^{e} =\int_{\Omega_{e}} \boldsymbol{B}_{a}^{T} \boldsymbol{\kappa} \boldsymbol{B}_{b} \quad a, b \in\left\{1,2, \cdots, N_{e n}\right\}\]

\(N_{en}\) is the number of element nodes.

The vector \(B_{a}\) has dimensions of \(N_{s d}\) by \(1,\) where
\(N_{s d}\) stands for number of spatial dimensions.

Integrals are usually restricted to individual elements and then the
contributions are added together through an assembly operation like that
of direct stiffness. In this case, the element stiffness matrix is given
by \[
k^{e}=\int_{\Omega_{e}} B^{T} D B
\] where the dimension of \(B\) is now \(3 \times 2 N_{e n}\) for 2 -D
elasticity problems - \(N_{e n}\) being the number of element nodes. In
3 -D the dimension of \(B\) becomes \(6 \times 3 N_{e n}\). The
dimension of \(k^{e}\) itself is either \(2 N_{e n} \times 2 N_{e n}\)
or \(3 N_{e n} \times 3 N_{e n} .\)

\hypertarget{nodal-forces}{%
\subsubsection{Nodal Forces}\label{nodal-forces}}

The element force vector is given by the following expression: \[
f^{e}=\int_{\Omega_{e}} N^{T} b+\int_{\Gamma_{i}^{e}} N^{T} \bar{t}
\]
\[\boldsymbol{F}=\int_{\Omega} \boldsymbol{N}^{T} \boldsymbol{b} d \Omega+\int_{\Gamma_{t}} \boldsymbol{N}^{T} \overline{\boldsymbol{t}} d \Gamma_{t}\]
where \(N\) will have dimensions \(2 \times 2 N_{e n}\) and
\(3 \times 3 N_{e n}\) for \(2-\mathrm{D}\) and \(3-\mathrm{D}\)
problems, respectively. {[}Note that the last integral in ( 18.9 ) is a
surface integral.{]}

For a 2D triangle: \[\boldsymbol{f}^{e}=\left(\begin{array}{c}
f_{11} \\
f_{21} \\
f_{12} \\
f_{22} \\
f_{13} \\
f_{23}
\end{array}\right)=\int_{\Gamma_{t}^{e}}\left[\begin{array}{cc}
N_{1}^{e} & 0 \\
0 & N_{1}^{e} \\
N_{2}^{e} & 0 \\
0 & N_{2}^{e} \\
N_{3}^{e} & 0 \\
0 & N_{3}^{e}
\end{array}\right]\left(\begin{array}{l}
0 \\
q
\end{array}\right) d \Gamma_{t}^{e}=\int_{\Gamma_{t}^{e}}\left(\begin{array}{c}
0 \\
0 \\
0 \\
N_{2}^{e} q \\
0 \\
N_{3}^{e} q
\end{array}\right) d \Gamma_{t}^{e}\]

An alternative and equivalent formula for computing such integrals that
is often used in continuum mechanics is \[
\int_{\Gamma} f(\boldsymbol{x}) \boldsymbol{n} d \Gamma=\int_{\gamma} f(\boldsymbol{x}(\boldsymbol{X})) \operatorname{det}\left[\frac{\partial \boldsymbol{x}}{\partial \boldsymbol{X}}\right]\left[\frac{\partial \boldsymbol{X}}{\partial \boldsymbol{x}}\right]^{T} \boldsymbol{n}_{X} d \gamma
\] where \(\boldsymbol{x}\) and \(\boldsymbol{X}\) are different
parameterizations of the same edge. \(\boldsymbol{x}\) are the physical
coordinates along the edge \(\Gamma\) and \(X\) are an alternative set
of coordinates along the edge \(\gamma=\boldsymbol{X}(\Gamma) ;\) in
continuum mechanics the coordinates \(\boldsymbol{X}\) are usually the
Lagrangian coordinates of the ``material points''; in our case they will
be the isoparametric coordinates. \(n\) is the normal to the edge in the
physical coordinates and \(n_{X}\) is the normal to the edge in the
\(X\) coordinates. For the case explored above this relation reduces to
\[
\int_{\Gamma} f(\boldsymbol{x}) \boldsymbol{n} d \Gamma=\int_{0}^{1} f\left(\boldsymbol{x}\left(t_{1}, 0,1-t_{1}\right)\right) \operatorname{det}\left[\frac{\partial \boldsymbol{x}}{\partial \boldsymbol{\xi}}\left(t_{1}, 0,1-t_{1}\right)\right]\left[\frac{\partial \boldsymbol{\xi}}{\partial \boldsymbol{x}}\left(t_{1}, 0,1-t_{1}\right)\right]^{T} \boldsymbol{n}_{X} d t_{1}
\] where \(\boldsymbol{n}_{X}\) is the normal to the edge in the parent
domain; in this case \(\boldsymbol{n}_{X}=(0,-1)^{T}\)

\hypertarget{isoparametric-formulations}{%
\subsection{Isoparametric
Formulations}\label{isoparametric-formulations}}

\[\int_{\Omega_{e}} f(\boldsymbol{x}) d \boldsymbol{x} \longmapsto \int f(\boldsymbol{x}(\boldsymbol{\xi})) \operatorname{det}\left[\frac{\partial \boldsymbol{x}}{\partial \boldsymbol{\xi}}\right] d \boldsymbol{\xi} \approx \sum_{\ell=1}^{N_{i n t}} W_{\ell} f\left(\boldsymbol{x}\left(\overline{\boldsymbol{\xi}}_{\ell}\right)\right) \operatorname{det}\left[\frac{\partial \boldsymbol{x}}{\partial \boldsymbol{\xi}}\left(\overline{\boldsymbol{\xi}}_{\ell}\right)\right]\]
where \(\bar{\xi}_{\ell}\) represent integration points and \(W_{\ell}\)
are the weights.
