\hypertarget{strain}{%
\section{Strain}\label{strain}}

There are many possible representations of deformation. Any reasonable
representation however, must be able to represent a rigid rotation of an
unstrained body without producing any strain. The engineering strain
fails here, thus it cannot be used for general geometrically nonlinear
cases \footnote{https://www.comsol.com/blogs/why-all-these-stresses-and-strains/}.

\hypertarget{green-lagrange}{%
\subsection{Green-Lagrange}\label{green-lagrange}}

\emph{The classical Lagrangian finite strain measure}\footnote{Felippa,
  Chp 7}

\begin{itemize}
\item
  In the definition of the Green-Lagrange strain tensor, all derivatives
  are with respect to the initial coordinates of the material particles.
  For this reason, we say that the strain tensor is defined with respect
  to the initial coordinates of the body. Also note that, although only
  up to quadratic terms of displacement derivatives appear in ( 6.55),
  this is the complete strain tensor; i.e., we have not neglected any
  higher-order terms \footnote{Bathe, 2014 978-0-9790049-5-7}.
\item
  contains derivatives of the displacements with respect to the original
  configuration \footnote{https://www.comsol.com/blogs/why-all-these-stresses-and-strains/}.
\item
  Values represent strains in material directions, similar to the
  behavior of the Second Piola-Kirchhoff stress. This allows a physical
  interpretation \footnote{https://www.comsol.com/blogs/why-all-these-stresses-and-strains/}.
\item
  Even for a uniaxial case, the Green-Lagrange strain is strongly
  nonlinear with respect to the displacement. If an object is stretched
  to twice its original length, the Green-Lagrange strain is 1.5 in the
  stretching direction. If the object is compressed to half its length,
  the strain would read -0.375 \footnote{https://www.comsol.com/blogs/why-all-these-stresses-and-strains/}.
\end{itemize}

\hypertarget{rotated-engineering}{%
\subsection{Rotated Engineering}\label{rotated-engineering}}
