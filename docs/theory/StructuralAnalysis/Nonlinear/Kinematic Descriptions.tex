\hypertarget{kinematic-descriptions}{%
\section{Kinematic Descriptions}\label{kinematic-descriptions}}

\hypertarget{total-lagrangian}{%
\subsection{Total Lagrangian}\label{total-lagrangian}}

\hypertarget{updated-lagrangian}{%
\subsection{Updated Lagrangian}\label{updated-lagrangian}}

\hypertarget{corotaional}{%
\subsection{Corotaional}\label{corotaional}}

\hypertarget{filippou-2016}{%
\subsubsection{Filippou 2016}\label{filippou-2016}}

The corotaional formulation postulates that the large displacement
kinematics of the element can be decomposed into the kinematics relative
to a reference frame that follows the element chord as it translates and
rotates with the relative end translations, and the rigid body
translation and rotation of this frame.
