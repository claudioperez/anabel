\hypertarget{section}{%
\section{}\label{section}}

4.3.2 Solution Process assumptions: 1) The applied loading consists only
of the reference load \(P_{r e f},\) which is increased monotonically by
incrementing the load factor \(\lambda\) until reaching the collapse
load factor \(\lambda_{c}\) 2) Plastic hinges that are ``open'' under a
load factor \(\lambda\) cannot ``close'' under a higher load factor:
this means that plastic hinge deformations increase monotonically under
the monotonically increasing reference load. \[
\boldsymbol{V}_{\boldsymbol{\varepsilon}}=\mathbf{F}, \boldsymbol{Q}_{c}+\boldsymbol{V}_{0}
\] The process for the determination of the last hinge to form consists
of the following steps: 1) Select any hinge of the collapse mechanism as
last to form and solve the kinematic relations for the corresponding
free dof displacements \(U_{f}^{t}\), where the superseript tr stands
for trial result. 2) Determine the plastic hinge deformations
\(V_{h p}^{\text {tr }}\) corresponding to \(U_{f}^{\text {tr }}\) with
\[
    \boldsymbol{V}=\boldsymbol{V}_{\varepsilon}+\boldsymbol{V}_{h p}^{t r}=\mathbf{A}_{f} \boldsymbol{U}_{f}^{t r}
    \] 3) If the sign of each plastic deformation matches the sign of
the corresponding basic force \(Q_{c}\) from the equilibrium equations,
the last hinge location is correct. The trial displacements and plastic
deformations from steps (1) and (2) give the free dof displacements
\(U_{f}\) and the plastic deformations \(\boldsymbol{V}_{h p}\) at
incipient collapse.

\begin{enumerate}
\def\labelenumi{\arabic{enumi})}
\setcounter{enumi}{3}
\tightlist
\item
  If the sign of one or more plastic deformations does not match the
  sign of the corresponding basic force, correct the free dof
  displacements and plastic deformations of Step (1) and (2) in a single
  step as described in the following.
\end{enumerate}

If the sign of one or more plastic deformations does not match the sign
of the corresponding basic
